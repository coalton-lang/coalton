\documentclass[12pt]{article}
\usepackage[]{todonotes}

\title{The Coalton Standard Library}

\begin{document}

\maketitle
\section{Introduction}
The Coalton standard library is a set of foundational packages that
are developed alongside the core language and share a common release
and versioning schedule. This report provides recommendations for the
scope, organization and content of these packages.

A principle: the core language should be as small as is reasonable,
and when possible, types should be defined in the standard library,
rather than as intrinsic to the core language. A more complex set of
criteria guide the distinction between standard library, and secondary
libraries.

\todo {naming} Neither 'secondary' nor 'contributed' libraries
captures the idea clearly.

The standard library should be sufficient to self-update.
\todo {update principle}

\section{Recommendations}

Discuss extended data syntax, with proposal to adopt EDN-style
notation for maps and vectors.
\todo {readable data syntax}

\subsection{General Recommendations}

\begin{itemize}
\item All docs should have a trivial but illustrative example with a
  known output.
\item Avoid ambiguity resulting from short nicknames, for instance,
  'ORD' for 'ORDER'
\item Avoid terse portmanteau prefixes such as "mk" or "to"
\end{itemize}

\todo{give example}

\subsection{Types}

\subsubsection{Primitive Types}

which types are core types? It may be the best that no types are
primitive types. What Coalton is really providing is abstraction.
\todo{primitive}

should both single and double floats be present in core? In which
languages does this matter? When is the past time that a programmer
reached for a short float? (Answer, recently, all the time: but
they've transcended the short float to demand hardware for bfloats,
and beyond)

\subsubsection{Tuples}

The ought be no need to characterize tuples as Tuple2, Tuple3,
Tupld4. At very least the numerically-suffixed structure classes that
provide tuple storage ought to be obscured by notation.

Tuples receive ordinal accessors (first, second, third).

It may be desirable to support named tuples, to provide better cues
about function.
\todo{question}
Is there a conflict or overlap with coalton-library/struct?

tuples as an existential type.
\todo{what did we mean?}

\subsection{Casting}

There are a lot of disparate casting functions that ought to be
replaced by into -- notable, as below, tointeger.

\subsection{Package Audit}

Recommendations:

\begin{itemize}
\item Make some of the packages under coalton-library secondary
  libraries. Criterion: anything that isn't used almost immediately in
  a moderately complex program, for instance, specialized kinds of
  math.
\item Replace ``coalton-library'' with ``coalton''.
\item Group concrete data types under ``coalton/data''.
\item Group typeclass definitions under ``coalton/class''.
\item Group system and host resource access under ``coalton/sys''.
\end{itemize}

\subsubsection{Existing Layout}

\begin{itemize}
\item coalton
\item coalton-library/classes
\item coalton-library/hash
\item coalton-library/types
\item coalton-library/builtin
\item coalton-library/functions
\item coalton-library/math/arith
\item coalton-library/math/bounded
\item coalton-library/math/fraction
\item coalton-library/math/integral
\item coalton-library/math/real
\item coalton-library/math/complex
\item coalton-library/math/elementary
\item coalton-library/math/dyadic
\item coalton-library/math/dual
\item coalton-library/bits
\item coalton-library/char
\item coalton-library/string
\item coalton-library/tuple
\item coalton-library/lisparray
\item coalton-library/optional
\item coalton-library/list
\item coalton-library/result
\item coalton-library/cell
\item coalton-library/randomaccess
\item coalton-library/vector
\item coalton-library/slice
\item coalton-library/hashtable
\item coalton-library/queue
\item coalton-library/monad/state
\item coalton-library/monad/free
\item coalton-library/iterator
\item coalton-library/ord-tree
\item coalton-library/ord-map
\item coalton-library/seq
\item coalton-library/system
\item coalton-library/file
\end {itemize}

\subsubsection{Proposed Layout}

\begin{itemize}
\item coalton
\item coalton/class
\item coalton/class/iterator
\item coalton/class/monad
\item coalton/data
\item coalton/data/array
\item coalton/data/byte
\item coalton/data/char
\item coalton/data/hash
\item coalton/data/list
\item coalton/data/map
\item coalton/data/mutable
\item coalton/data/queue
\item coalton/data/seq
\item coalton/data/string
\item coalton/data/tuple
\item coalton/data/tree
\item coalton/data/vector
\item coalton/function
\item coalton/math
\item coalton/sys
\item coalton/sys/file
\end {itemize}

Note the idea to merge slice and array
\todo{note}

FIXME: which existing packages are in each category?

\subsubsection{coalton-library/math/fraction}

R: mkfraction - the 'mk' prefix is unique here: is there a naming convention to apply?

Q: Fraction is a primitive type, just a wrapper for a lisp fraction, and appears in primitive-types.

N: implementation of primitive types is kind of a mess, for instance
the defvar in typechecker/types.

the type definition should probably be in this file

clearly mark that this is a CL rational

\subsubsection{coalton-library/tuple}

\section{Hashing}

do other langauges "Expose an interface to hashing"

\section{ADTs}

loading array into system - doesn't need a tmpfile

write a whole para on traverse. the methods all need descriptions

talk about traversable and sequence

there should be a simple syntax for defining new byte types

\subsubsection{coalton-library/classes}

All of the stuff about tuples should be defined in the same file. Some
is in classes, somme is in tuple.

The statement 'heterogeneous collection of itsm' is inaccurate.

What is the core stuff that belongs in classes

Note: there should be some way to easily define a bit or byte type - for U7, or whatever particular value.

N: rationalization of max and min

\subsubsection{coalton-library/math/integral}

Q: tointeger seems to overlap completely with (into Integer ..)
- can it be collapsed? What is the purpose, otherwise?

Q: "to-" prefix: is it like "mk-" prefix?

N: it may be confusing to have multiple (semantically identical)
definitions

Q: Are there problems with integral->num? Can it be killed? It seems
ot only add syntax, in that it could be equally well expressed as
(into X whatever) rather than (the foo (integral->num whatever)).

Q: How sound is semigroup? What is it used for, what does it support?

R: HASH should be a nonexternal library: just provides one method to hash table.

HASH should be in hashtablr files.

There should be documentation here that proivdes a rationale for a hash user.

\subsubsection{coalton-library/builtin}

Q can the file builtin go away?

undefined can go to functions, the others can all go to boolean.

\subsubsection{coalton-library/functions}

Q /= goes to classes

Q can flip be gotten rid of

Any time there is some arbitrary haskell thing added to a part of the
standard library, there ought to be an example.

Question: trace and trace object need ot be move to format library.

trace

R: The proxy system (or its interface) should be reassessed. Consider
the example of setting the element type when reading a vector.

Q: Booolean: when would you use Ord over Boolean?

R: 'builtin' is chop suey of boolean and debugging: tweeze

Q: are there different 'flavors' of Boolean that should be picked apart?

Q: is there a way to eject builtin types from the typecheckers?

Q:  How much does the typechecker need ot know about primitive types at compile time?

R: "/=" should be in classes, not in fucntions

Question: we need a thin wrapper on lisp format.

Question: format error function

Recommendation: Move dyadic and dual to secondary packages.

Note: bifunctor shouldn't in core, someone can define it as part of
      something they need

There needs to be some docstring for SEQUENCE [FUNCTION] - switches between types of containers

Unwrappable should be its own thing, and the gateway to the condition
system, since it's a way to handle errors.

Optional and result should also go into that file.

The repr for enum types should be exposed in the documentation.

The documentation needs ot be either in file or alphabetical order.

\subsubsection{coalton-library/math/arith}

Note:- bits documentation is outright wrong about the twos complement stuff

Question: define char at the top of this file, not as an 'early type'

    - same for 'string'

nothing needs a char until the car file!

Note: docs should be generated in alphabetical order (or file order!)

Splin N string - > tuple string

Note: unchecked -> unsafe

Question: documentation for all of the macros

Question: where is the documentation for define-struct, define-class, define-instance!

QQ And the lisp form.

Note: car/cdr -> first and rest

NN remanticsd of remove-if are weird - only operates on  fiorts match?

Question: equivalence-classes?

Note: audit fpor "XXX"! with return value + side effects or mutation

Question: It seems like 'cell' should be a pure (atomic?) mutation interface, that doesn'tr have any assumptions about the kind of type to which a reference is being maintained.

Question: is random-access like this? Is r-a something that is stored in cell?

Note: are vectors able to supersede a bunch of other stuff (cell, list)

"implemented by a linked list - what?

Question: Free monad - why are the type variables renamed in the documentation?

NN up-to and down-from are weird.

Question: elementwise-hash! is specialized to the implementation of ord-trees, and belongs there

Question: remove-duplicates! shows up in a weird place and duplicates a documentation section

Note: ord-tree collect! tree - should be into iterator?

Question: ".like replace-or-insert" - do those really have to have names that are so similar?

Q - ord-map - why is there not ord in entries?

R: The SEQ types seems hollowed out or superseded by vector and iterator stuff: does it stay or go?

N: In general, more consistency for foundational ADTs should be sought so that the available types conform to common understanding of roles and complexity classes.

  - map
  - vector
  - list
  - cons

Also, mutable and immutable structures should be distinguished

R: Replace -unsafe with -safe?

bounded (minvalue maxvalue) is kind of like hash in that i'ts ring-0 or lower order

\subsubsection{coalton-library/bits}

proposal: Call it 'byte' instead.

DO dpb and variable width bitfield operations go into bits?

ascii - really means utf8 or whatever the platform uses

the char package needs to have all of its assumptions checked

\end{document}
